\chapter{Conclusioni}
In questo lavoro abbiamo analizzato la possibilità di inferire la struttura di una rete bayesiana in grado di fornire un valido supporto ai medici nell'identificazione delle cause della sindrome del \textit{BlueBaby}. Sono state confrontate due differenti algoritmi di apprendimento, e la nostra scelta è ricaduta su \texttt{Tabù Search}, in quanto in grado di ottenere soluzioni migliori. Abbiamo poi valutato le varie funzioni di score disponibili, identificando \texttt{k2} come la soluzione in grado di ottenere un risultato che approssimasse in modo ottimo la rete proposta dal paper.
Sono state poi valutate le performance di predizione del valore di \textit{Disease} in caso di un'evidenza completa o parziale della rete, ottenendo performance statisticamente equivalenti. Abbiamo poi provato ad invertire la struttura della rete del paper, capovolgendo le relazioni di dipendenza causale: ciò ha comportato una riduzione delle performance e un'aumento del numero di genitori medio dei nodi, che si è tradotto in un drastico aumento del numero di casi da valutare nelle CPT.

Sono state poi ricercate strutture all'interno della rete che garantissero la d-separazione tra il nodo \textit{Disease} e gli altri nodi della rete. Da questa analisi è emerso che l'unica possibilità di indipendenza all'interno della rete è assoggettata alla conoscenza del valore di alcune situazioni patologiche che è impossibile rilevare se non mediate esami mirati, che risultano essere i figli di tali nodi. Risulta quindi priva di significato l'idea di possedere evidenza di tali nodi ma non degli esami che sono in gradi di accertarli. Da ciò consegue che non è possibile eliminare alcun esame al fine di una corretta stima della patologia che causa la sindrome trattata.
Analizzando poi la distribuzione della realizzazione della variabile \textit{Disease} al variare dell'evidenza fornita, abbiamo rilevato come alcuni valori degli esami \textit{XrayReport} e \textit{LVHreport} siano in grado di influenzare il valore del nodo in analisi, portandolo a realizzarsi come una determinata malattia oltre il $60\%$ delle volte.

Al fine di garantire un supporto utile ed immediato ai medici, è stata realizza una demo interattiva, che permette all'utente di inserire le evidenze a sua disposizione e di vedere come varia la distribuzione di realizzazione del nodo \textit{Disease}, andando di fatto ad individuare la causa più probabile del colore cianotico del neonato.
